% Parametri che modificano il file main.tex
% Le uniche parti da cambiare su main.tex sono:
% - vari \vspace tra sezioni
% - tabella azioni da intraprendere
% - sezione altro

\def\data{2024-04-12}
\def\oraInizio{16:00}
\def\oraFine{16:30}
\def\luogo{Discord}

\def\tipoVerb{Interno} % Interno - Esterno

\def\nomeResp{Dugo A.} % Cognome N.
\def\nomeVer{Bresolin G.} % Cognome N.
\def\nomeSegr{Dugo A.} % Cognome N.

\def\firmaResp{alberto.png} % nome Responsabile

\def\listaPartInt{
Bresolin G.,
Campese M.,
Ciriolo I.,
Dugo A.,
Feltrin E.,
Michelon R.
}


\def\listaRevisioneAzioni {
{Revisione documento Piano di progetto;},
{Revisione documento Piano di qualifica;},
{Revisione documento Norme di progetto;},
{Revisione documento Manuale utente;}
}

\def\listaOrdineGiorno {
{Autovalutazione colloquio con professor Cardin;},
{Aggiornamento del team sulla presentazione in data da definire.}
}



\def\listaDiscussioneInterna {
{Il team si è suddiviso le slide della presentazione della seconda revisione Product Baseline con il professor Vardanega, garantendosi che ciascun membro abbia chiara comprensione di cosa esporre;},
{Il team ha discusso su chi dovesse contattare il gruppo Code Cowboys.}
}


% Parametri che modificano il file main.tex
% Le uniche parti da cambiare su main.tex sono:
% - vari \vspace tra sezioni
% - tabella azioni da intraprendere
% - sezione altro

\def\data{2024-04-12}
\def\oraInizio{16:00}
\def\oraFine{16:30}
\def\luogo{Discord}

\def\tipoVerb{Interno} % Interno - Esterno

\def\nomeResp{Dugo A.} % Cognome N.
\def\nomeVer{Bresolin G.} % Cognome N.
\def\nomeSegr{Dugo A.} % Cognome N.

\def\firmaResp{alberto.png} % nome Responsabile

\def\listaPartInt{
Bresolin G.,
Campese M.,
Ciriolo I.,
Dugo A.,
Feltrin E.,
Michelon R.
}


\def\listaRevisioneAzioni {
{Revisione documento Piano di progetto;},
{Revisione documento Piano di qualifica;},
{Revisione documento Norme di progetto;},
{Revisione documento Manuale utente;}
}

\def\listaOrdineGiorno {
{Autovalutazione colloquio con professor Cardin;},
{Aggiornamento del team sulla presentazione in data da definire.}
}



\def\listaDiscussioneInterna {
{Il team si è suddiviso le slide della presentazione della seconda revisione Product Baseline con il professor Vardanega, garantendosi che ciascun membro abbia chiara comprensione di cosa esporre;},
{Il team ha discusso su chi dovesse contattare il gruppo Code Cowboys.}
}


\def\listaDecisioni{
{Il team ha deciso di inviare mail al professor Vardanega in merito alla seconda revisione della Product Baseline dopo aver ricevuto l'esito del semaforo verde del professor Cardin;},
{La suddivisione della presentazione avviene in questo modo:\\
    \textbf{Orlandi G.} : Introduzione.\\
    \textbf{Campese M.} : Stato di completamento corrente del prodotto in rapporto alle attese, per progettazione, codifica, verifica;\\
    \textbf{Michelon R.} : Propria auto-valutazione dell'esito del colloquio PB svolto con il docente Cardin, del riscontro MVP del proponente;\\
    \textbf{Ciriolo I.} : Riscontro MVP dei risultati di apprendimento conseguiti;\\
    \textbf{Bresolin G.} : Miglioramenti attuati nella qualità del proprio way of working e nella gestione di progetto; \\
    \textbf{Dugo A.} : Difficoltà incontrate e loro mitigazione;\\
    \textbf{Feltrin E.} : Consuntivo di impegno individuale e collettivo, aggiornato alla data corrente, inclusi totali complessivi individuali;
},
{Michelon R. si occuperà di contattare il gruppo Code Cowboys per accordarsi sulla suddivisione delle slide che il team SWEetCode dovrà esporre durante il meeting nella sede di AzzurroDigitale.},
}