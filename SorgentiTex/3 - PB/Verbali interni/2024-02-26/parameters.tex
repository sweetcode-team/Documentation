% Parametri che modificano il file main.tex
% Le uniche parti da cambiare su main.tex sono:
% - vari \vspace tra sezioni
% - tabella azioni da intraprendere
% - sezione altro

\def\data{2024-02-26}
\def\oraInizio{11:30}
\def\oraFine{12:30}
\def\luogo{Discord}

\def\tipoVerb{Interno} % Interno - Esterno

\def\nomeResp{Ciriolo I.} % Cognome N.
\def\nomeVer{Michelon R.} % Cognome N.
\def\nomeSegr{Dugo A.} % Cognome N.

\def\nomeAzienda{AzzurroDigitale}
\def\firmaAzienda{azzurrodigitale.png}
\def\firmaResp{irene.png} % nome Responsabile

\def\listaPartInt{
Bresolin G.,
Campese M.,
Ciriolo I.,
Dugo A.,
Feltrin E.,
Michelon R.,
Orlandi G.
}

\def\listaPartEst{
Davanzo C.,
Bendotti E.
}

\def\listaRevisioneAzioni 
{Revisione della progettazione logica e di dettaglio svolte durante lo sprint.}


\def\listaOrdineGiorno {
{Stilare la retrospettiva dello sprint concluso in data odierna;},
{Formulare alcuni quesiti da porre ad Azzurrodigitale emersi durante la fase di progettazione;},
{Formulare alcuni quesiti da porre al professor Cardin R. emersi durante la fase di progettazione;},
{Assegnazione di task per lo sprint appena iniziato.}
}


\def\listaDiscussioneInterna {
{Il team sta attualmente esplorando come permettere modifiche alla configurazione del modello LLM, inclusi il vectordatabase e il modello di embeddings.
\\Ciò significherebbe dare all’utente (l’operaio) la libertà di modificare sia il modello di embeddings che il vectorstore.
Tuttavia, ci sono alcune questioni importanti che il team deve porre all’azienda: \\
Ogni volta che si modifica il modello di embedding, è necessario rigenerare il vectorstore. Questo perché ogni modello di embedding produce embeddings di dimensioni diverse, il che comporterebbe la perdita di tutti gli embeddings generati in precedenza. 
Una domanda importante da porre all'azienda è se lasciare all’utente la libertà di scelta.\\
Una soluzione potrebbe essere quella di affidare la gestione della configurazione a un utente esperto, come l’operatore responsabile dell’installazione e della manutenzione del software.\\ 
Per quanto riguarda la modifica del vectorstore, il team dovrebbe considerare la possibilità di migrare gli embeddings. Tuttavia, questa non è un’operazione economica né veloce. Anche in questo caso, il team ritiene che la decisione dovrebbe essere presa dall’operatore esperto.
Queste domande saranno un punto di discussione per il prossimo incontro interno con l’azienda;},
{Il team ha inoltre affrontato l'upload e l'eliminazione di uno o più documenti, essendo questa fase composta da due sottofasi: \\
\begin{itemize}
    \item Caricamento o eliminazione documento/i su AWS3;
    \item Caricamento/eliminazione embeddings sul vector store.
\end{itemize}
},
{Inoltre il team ha trattato una domanda emersa da porre al professor Cardin tramite mail, la domanda riguarda l'utilizzo dei manager per gestire l'interazione con strumenti esterni al nostro applicativo.}
}



\newcommand{\domris}[2]{\textbf{#1}\\#2}

\def\listaDiscussioneEsterna {
}

\def\listaDecisioni {
{Il team ha deciso di mandare una mail al professor Cardin per risolvere il dubbio relativo ai manager;},
{Il team si informerà e parlerà con l'azienda per risolvere i dubbi emersi.}
}