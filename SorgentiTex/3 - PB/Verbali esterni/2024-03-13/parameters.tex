% Parametri che modificano il file main.tex
% Le uniche parti da cambiare su main.tex sono:
% - vari \vspace tra sezioni
% - tabella azioni da intraprendere
% - sezione altro

\def\data{2024-03-13}
\def\oraInizio{17:00}
\def\oraFine{17:30}
\def\luogo{Google Meet}

\def\tipoVerb{Esterno} % Interno - Esterno

\def\nomeResp{Bresolin G.} % Cognome N.
\def\nomeVer{Ciriolo I.} % Cognome N.
\def\nomeSegr{Campese M.} % Cognome N.

\def\nomeAzienda{AzzurroDigitale}
\def\firmaAzienda{azzurrodigitale.png}
\def\firmaResp{gianluca.png} % nome Responsabile

\def\listaPartInt{
Bresolin G.,
Campese M.,
Feltrin E.,
Michelon R.,
Orlandi G.
}

\def\listaPartEst{
Bendotti E.,
Davanzo C.,
}

\def\listaRevisioneAzioni {{Il team ha esposto ad AzzurroDigitale la codifica del backend realizzata, non ottenendo alcun feedback.}}

\def\listaOrdineGiorno {
{Revisione della codifica backend eseguita dal team;},
{Aggiornamento previsioni requisiti opzionali che verranno accolti dal team;},
{Aggiornamento pianificazione futura dello sprint corrente}.
}


\def\listaDiscussioneInterna {
{Discussione 1;},
{Discussione 2;},
}

\newcommand{\domris}[2]{\textbf{#1}\\#2}

\def\listaDiscussioneEsterna {
\domris
{In merito alla codifica del backend effettuata avete qualche feedback da darci?}
{L'azienda ha precisato che senza alcun relativo componente di frontend sviluppato non risulta per ora possibile fornire alcun feedback. Sarebbe infatti da preferire uno sviluppo parallelo tra backend e frontend in modo da poter fornire al 'cliente' del materiale consultabile direttamente;},
\domris
{Aggiornamento requisiti opzionali che il team ritiene di poter soddisfare:}
{Il team si pone l'obiettivo inizialmente di soddisfare tutti i requisiti obbligatori e, solo in seguito prendere in considerazione lo sviluppo di componenti che soddisfino requisiti accordati come opzionali o desiderabili.
Nonostante ciò, il team ritiene che l'implementazione della funzionalità di interazione vocale con il chatbot sia fattibile con i tempi di consegna previsti e con le risorse del progetto pianificate;},
\domris
{Cosa prevede la pianificazione del seguente sprint?}
{Il team nel presente sprint ha pianificato la codifica delle componenti di frontend e relativi test, in modo da ottenere nel prossimo incontro un feedback utile sulla codifica svolta. }
}




\def\listaDecisioni {
{Il team nel prossimo sprint pianifica la codifica delle componenti di frontend.}
}