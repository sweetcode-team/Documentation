% Parametri che modificano il file main.tex
% Le uniche parti da cambiare su main.tex sono:
% - vari \vspace tra sezioni
% - tabella azioni da intraprendere
% - sezione altro

\def\data{2024-03-26}
\def\oraInizio{17:00}
\def\oraFine{18:00}
\def\luogo{Google Meet}

\def\tipoVerb{Esterno} % Interno - Esterno

\def\nomeResp{Dugo A.} % Cognome N.
\def\nomeVer{Bresolin G.} % Cognome N.
\def\nomeSegr{Dugo A.} % Cognome N.

\def\nomeAzienda{AzzurroDigitale}
\def\firmaAzienda{azzurrodigitale.png}
\def\firmaResp{alberto.png} % nome Responsabile

\def\listaPartInt{
Bresolin G.,
Campese M.,
Feltrin E.,
Michelon R.,
Orlandi G.
}
\def\listaPartEst{
Bendotti E.,
Davanzo C.,
}

\def\listaRevisioneAzioni {{Il team ha codificato tutte le componenti del frontend come da previsione.}}

\def\listaOrdineGiorno {
{Esposizione del Minimum Viable Product;},
{Feedback da parte di AzzurroDigitale rispetto all'esposizione sopra citata}.
}


\newcommand{\domris}[2]{\textbf{#1}\\#2}



\def\listaDiscussioneEsterna {
\domris
{Quali difficoltà ha incontrato il gruppo durante lo sviluppo dell’MVP?
}{La difficoltà maggiore che il team ha riscontrato riguarda lo sviluppo dei test, questo perché era  un ambiente nuovo e mai esplorato dal gruppo stesso.s
Il team non ha riscontrato ulteriori difficoltà;},
\domris
{come avete organizzato il lavoro? Avete utilizzato qualche software?
}{il team ha utilizzato Postman per simulare le chiamate dal front end. Inoltre a causa di un limite di richieste, posto dall’account gratuito, ad AWS il team ha utilizzato dei mock i quali simulavano le risposte di AWS in modo tale da non sprecare richieste;
},
\domris{Per quanto riguarda i requisiti sono stati soddisfatti tutti?
}{Sì, sono stati soddisfatti tutti i requisiti obbligatori ed inoltre è stato soddisfatto un requisito opzionale il quale il team ha ritenuto importante soddisfarlo:
inserimento messaggio vocale 
è possibile trovare il soddisfacimento del requisito nella chat.
}
}

\def\listaDecisioni {
{l'MVP è stato accettato da AzzurroDigitale;},
{Incontro in sede AzzurroDigitale in corrispondenza dell'evento "QMeeting" nel quale il team dovrà esporre allo staff di AzzurroDigitale l'MVP e le difficoltà incontrate durante la sua progettazione e codifica.}
}