% Parametri che modificano il file main.tex
% Le uniche parti da cambiare su main.tex sono:
% - vari \vspace tra sezioni
% - tabella azioni da intraprendere
% - sezione altro

\def\data{2023-11-15}
\def\oraInizio{15:00}
\def\oraFine{16:00}
\def\luogo{Piattaforma Google Meet}

\def\tipoVerb{Esterno} % Interno - Esterno

\def\nomeResp{Feltrin E.} % Cognome N.
\def\nomeVer{Campese M.} % Cognome N.
\def\nomeSegr{Orlandi G.} % Cognome N.

\def\nomeAzienda{AzzurroDigitale}
\def\firmaAzienda{azzurrodigitale.png}
\def\firmaResp{emanuele.png} % nome Responsabile

\def\listaPartInt{
Bresolin G.,
Campese M.,
Ciriolo I.,
Dugo A.,
Feltrin E.,
Michelon R.,
Orlandi G.
}

\def\listaPartEst{
\textit{AzzurroDigitale:},
Caliendo G.,
Davanzo C.}



\def\listaRevisioneAzioni {x}

\def\listaOrdineGiorno {
{Riunione con l'azienda AzzurroDigitale;},
{Breve introduzione e raccolta requisiti generali di progetto per una stesura iniziale dell'Analisi dei requisiti;},
{Possibilità di porre alcune domande generiche sul progetto ai referenti Caliendo G. e Davanzo C., con seguente approfondimento dei vari temi toccati.\\}
}

\def\listaDiscussioneInterna {
{Discussione 1;},
{Discussione 2;},
}

\newcommand{\domris}[2]{\textbf{#1}\\#2}

\def\listaDiscussioneEsterna {
\domris
{In generale, quali sono i requisiti funzionali fondamentali che deve avere il sistema?
}
{Certamente la ricezione in input dei documenti e la gestione degli stessi in modo efficiente;
},
\domris
{Quali funzionalità principali dovremmo rendere disponibili per il caricamento dei documenti?
}
{Sicuramente il sistema dovrà essere capace di ricevere in input ed elaborare documenti \textit{PDF}; rientrerebbe tra i requisiti desiderabili anche il caricamento di documenti \textit{Word} e, nel caso il prodotto prendesse piede, il caricamento massivo di documenti;
},
\domris
{I documenti caricati dovranno rimanere accessibili?
}
{L'azienda ci ha risposto che i documenti caricati devono rimanere accessibili, anche con un semplice catalogo paginato;
},
\domris
{Come deve procedere il sistema se l’utente vuole rimuovere un documento dal materiale con cui il modello è stato addestrato?  Il modello dovrà essere riaddestrato?
}
{L'azienda conferma che se un manuale dato al sistema subisce modifiche o viene eliminato, a fronte di una evoluzione del macchinario a cui si riferisce, il sistema dovrà evolvere ed essere riaddestrato;
},
\domris
{Vi sono altre specifiche riferite all'interazione tra l'utente e il \textit{chatbot}?
}
{L'utente dovrà poter chattare con molta semplicità e ricevere risposte pertinenti basate sui documenti caricati a sistema, tuttavia, all'utente non sarà permesso fare domande non attinenti al proprio lavoro (ad esempio: "Quali sono le previsioni meteo per domani?"). Un requisito che interessa particolarmente alla azienda è la possibilità di comunicare a voce con il sistema e di ricevere output vocali. Non è necessario poter aprire più chat contemporaneamente ma è desiderabile avere uno storico delle cose chieste.}.
}





\def\listaDecisioni {x}